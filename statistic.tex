%% -----------------preamble------------------ %%

%% define the document as an article,
%% size of a4paper, twosided and with two columns
\documentclass[twoside, a4paper, twocolumn]{article}

%% use the utf-8 encoding for the input
\usepackage[utf8]{inputenc}

%% diagrams
\usepackage{pgfplots}

%% cache diagrams -> faster compile times
\usepgfplotslibrary{external}
\tikzexternalize

%% math env and symbols
\usepackage{amsmath}

%% math more symbols
\usepackage{amssymb}

%% hyperlinks
\usepackage[hidelinks]{hyperref}

%% 
\usepackage[small]{titlesec}

%% include everything until subsections in the toc
\setcounter{tocdepth}{5}
\setcounter{secnumdepth}{5}
%% ------------------------------------------- %%

\begin{document}
    %% * suffix prints the section without the counter prefix
    \section*{Statistics - Exam preparation}
    %% italic 
    by \textit{xnacly}

    \tableofcontents

    %% sections arent prefixed with a counter
    \section{Introduction}
    Statistics is a fairly big field. Therefore this paper will only include
    the absolutely necessary topics for passing the university class.

    \section{Abstract}
    This paper starts of with symbols used in the field of statistics, their
    meaning and in what context they are commonly used. Following Combinatorics
    is thematized.

    \section{Symbols and special characters}

    %% env to list items, similar to the <ul> HTML tag 
    \begin{itemize}
       %% anything between $ and $ is interpreted as a mathblock
       \item $n!$ Faculty / Fakultät
       \item $\binom{n}{k}$ Binomial Coefficient / Binomialkoeffizient
       \item $\Omega$ Event set / Ergebnismenge
       \item $\omega$ Result / Ergebnis
       \item $A \subseteq \Omega$ Event / Ereignis
       \item $\{\omega\}$ Elementary event / Elementarereignis
       \item $\mathbb{P}$ Probability measure / Wahrscheinlichkeitsmaß
       \item $\mathbb{P}(A)$ Event propability / Wahrscheinlichkeit eines Ereignisses
       \item $\mathbb{E}(X), \mu_x, \mu$ Expected value / Erwartungswert
       \item $\sigma$ Deviation from the mean / Standardabweichung
       \item $\mathrm{Var}(X), \sigma^2_x$ Variance / Varianz
       \item $\mathrm{Cov}(X,Y), \sigma_{XY}$ Kovarianz von X und Y
       \item $\mathcal{N}(\mu, \sigma^2)$ Normal distribution / Normalverteilung
       \item $\varphi$ Bell curve / Glockenkurve
       \item $\Phi$ Error function / Fehlerintegral
       \item $X$ Random variable / Zufallsvariable
       \item $\mathrm{Z}$ Standard score / standard-normalverteilte
           Zufallsvariable
           \footnote{read more: \href{https://en.wikipedia.org/wiki/Standard_score}{wikipedia}}
       \item $\textrm{Bin}(n, p)$ Binomial distribution / Binomialverteilung
       \item $\textrm{Pois}(\lambda)$ Poisson distribution / Poission-Verteilung
       \item $\textrm{Exp}(\lambda)$ Exponential distribution / Exponentialverteilung
    \end{itemize}

    \section{Combinatorics}
    This chapter will introduce the Binomial Coefficient, Faculty, Pascal's
    triangle and the binomial Theorem.

    \subsection{Binomial Coefficient}
    \textit{n choose k}; used to calculate the Amount of Sets in
    $\{1,\textrm{...},n\}$ with exactly $k$ Elements. $n$ needs to be positive
    and $k$ and $n$ have to meet the following criteria: $n \in \mathbb{N}, 0
    \leq k \leq n$.
    \begin{equation}
        \binom{n}{k} = \frac{n!}{k! \cdot (n-k)!}
    \end{equation}
    Choosing $k$ different Numbers from $\{1,\textrm{...},n\}$, there are $n$
    possibilities for the first number, $n-1$ possibilities for the second, and
    so forth. 

    \subsection{Faculty}
    This can be described with $n!$. The faculty is defined as the
    product of decrementing $n$ by an increasing subtrahend:
    \begin{equation}
        n! := n \cdot (n-1) \cdot (n-2) \textrm{...}
    \end{equation}
    If $n = k$, there are $n!$ possibilities to choose $k$ Elements from $n$.
    $0! = 1$.

    \subsection{Pascal's triangle}
    The pascal's triangle can be used to visualize the binomial
    coefficient.\footnote{read more about \textit{pascal's triangle}
    here: \href{https://en.wikipedia.org/wiki/Pascal's_triangle}{wikipedia}}

    \begin{equation}
        \binom{n+1}{k+1} = \binom{n}{k} + \binom{n}{k+1}
    \end{equation}

    \subsection{Binomial Theorem}
    Allows for expressing the exponents of $(x+y)^n, n \in \mathbb{N}$ as a
    polynomial with the degree of $n$.
    \begin{equation}
        (x+y)^n = \sum^n_{k=0} \binom{n}{k} \cdot x^k \cdot y^{n-k}
    \end{equation}

    \section{Probability theory}
    This chapter contains information on how to calculate probabilites.

    \subsection{Event set}
    The set containing results of the \textit{experiment} $E$ is notated via
    the \textit{event set} ($\Omega$). Sub sets of $\Omega$ are
    \textit{events} ($\omega$). \textit{Events} with one entry are \textit{elementary
    events} $\{\omega\}$.
    If $\Omega$ is finite: $\forall \omega \in \Omega,\mathbb{P}(\omega) \geq 0$. 
    The sum of all propabilities of $\omega \in \Omega$ is $1$\footnote{$\sum_{\omega
    \in \Omega} \mathbb{P}(\omega) = 1$}.

    \subsection{Random variable}
    $X: \Omega \rightarrow \mathbb{R}$ we define $\{X = x\} := \{\omega |
    X(\omega) = x\}$ and can therefore shorten our definition of the
    propability that $X$ is $x$ to: $P(X = x) := P(\{X = x\})$
    \begin{eqnarray}
        x \rightarrow \mathbb{P}(X = x) \\
        x \rightarrow \mathbb{P}(X \leq x)
    \end{eqnarray}

    The equation 5 defines the density / probability function of $X$ and the
    equation 6 the distribution function of $X$.
\end{document}
